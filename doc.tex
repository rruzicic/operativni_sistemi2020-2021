%&pdflatex
\documentclass{article}
\usepackage[a4paper, total={200mm, 210mm}]{geometry}

\begin{document}

\textbf{Operativni sistemi(RA2020/2021) - odgovori iz knjige}
\section{Uvod}
\enumerate{}
\item Koje poslove obavlja operativni sistem? \\
-Operativni sistem je program koji objedinjuje u skladnu celinu raznorodne delove
računara i sakriva od korisnika one detalje funkcionisanja ovih delova koji nisu bitni za
korišćenje računara.
\item Šta obuhvata pojam datoteke?\\
-Pojam datoteke obuhvata sadržaj i atribute datoteke. Sadržaj datoteke predstavljaju
korisnički podaci. U atribute datoteke spada, na primer, veličina datoteke ili vreme
njenog nastanka.
\item Šta se nalazi u deskriptoru datoteke?\\
-Atributi datoteke, tj. vreme nastanka i veličina datoteke.
\item Šta omogućuju datoteke?\\
-Datoteke su namenjene za trajno čuvanje podataka. Pristup ovim podacima se svodi
na čitanje i pisanje sadržaja datoteka.
\item Šta obavezno prethodi čitanju i pisanju datoteke?\\
-Prethodi njeno otvaranje, radi pripreme zahtevanog pristupa podacima(npr. prebacivanje
deskriptora datoteke iz masovne memorije u radnu memoriju).
\item Šta sledi iza čitanja i pisanja datoteke?\\
-Iza čitanja i
pisanja sadržaja datoteke obavezno sledi njeno zatvaranje (na primer, radi prebacivanja
podataka i deskriptora datoteke iz radne memorije u masovnu memoriju)
\item Šta obuhvata pojam procesa?\\
-Pojam procesa obuhvata aktivnost, sliku i atribute procesa.
\item Šta se nalazi u deskriptoru procesa?\\
-Atributi procesa, u atribute procesa spadaju, na
primer, stanje procesa i njegov prioritet.
\item Koja stanja procesa postoje?\\
-"aktivan", "čeka" i "spreman"
\item Kada je proces aktivan?\\
-Proces je aktivan kada procesor izvršava program.
\item Šta je kvantum?\\
-Ako postoji nekoliko procesa najvišeg (istog) prioriteta, tada je
bitna ravnomerna raspodela procesorskog vremena izmedju njih. Ona se postiže ako aktivni proces prepušta procesor spremnom procesu istog (najvišeg) prioriteta čim
istekne unapred odredjeni vremenski interval. Ovaj interval se naziva kvantum
(quantum).
\item Šta se dešava nakon isticanja kvantuma?\\
-Prekida se sat, obrada ovakvih
prekida sata izaziva prevodjenje aktivnog procesa u stanje "spreman" i preključivanje
procesora na onaj od ostalih spremnih procesa najvišeg prioriteta koji je najduže u stanju
"spreman".
\item Po kom kriteriju se uvek bira aktivan proces?\\
-Kriterijum za biranje aktivnog procesa predstavlja prioritet procesa.
\item Koji prelazi su mogući izmedju stanja procesa?\\
- Mogući prelazi: ČEKA $\rightarrow$ SPREMAN, SPREMAN $\rightarrow$ AKTIVAN,
 AKTIVAN $\rightarrow$ SPREMAN, AKTIVAN $\rightarrow$ ČEKA
\item Koji prelazi nisu mogući izmedju stanja procesa?\\
- SPREMAN $\rightarrow$ ČEKA, ČEKA $\rightarrow$ AKTIVAN
\item Šta omogućuju procesi?\\
-Procesi omogućuju bolje iskorišćenje računara (procesora) i njegovu bržu reakciju na
dešavanje vanjskih dogadjaja.
\item Šta karakteriše sekvencijalni proces? \\
-Proces je sekvencijalan ako je njegov trag
odredjen u vreme programiranja, odnosno ako zavisi samo od obradjivanih podataka.
\item Šta karakteriše konkurentni proces?\\
-Procesi sa više istovremeno (concurrently) postojećih niti se nazivaju konkurentni
procesi.
\item Šta ima svaka nit konkurentnog procesa?\\
-Svaka nit procesa ima svoj prioritet, svoje
stanje, svoj stek, pa i svoj deskriptor.
\item Koju operaciju uvodi modul za rukovanje procesorom?\\
-Operaciju preključivanja.
\item Po čemu se razlikuju preključivanja izmedju niti istog procesa i preključivanja
izmedju niti raznih procesa?\\
-Ključna razlika izmedju niti koje pripadaju istom procesu i niti
koje pripadaju raznim procesima je da su prve niti u adresnom prostoru istog procesa (da
bi mogle da saradjuju), dok su druge niti u adresnim prostorima raznih procesa.
\item Koje operacije uvodi modul za rukovanje kontrolerima?\\
-(drajverske) operacije ulaza i izlaza.
\item Šta karakteriše drajvere?\\
-Drajveri imaju zadatak da konkretan ulazno-izlazni uredjaj predstavi u
apstraktnom obliku sa jednoobraznim i pravilnim načinom korišćenja.
\item Koje operacije uvodi modul za rukovanje radnom memorijom?\\
-operacije zauzimanja i oslobadjanja
\item Koje operacije poziva modul za rukovanje radnom memorijom kada podržava
virtuelnu memoriju?\\
-operacije ulaza i izlaza(zbog prenosa podataka izmedju radne i masovne memorije).
\item Koje operacije uvodi modul za rukovanje datotekama?\\
-operacije otvaranja, zatvaranja, čitanja i pisanja
\item Koje operacije poziva modul za rukovanje datotekama?\\
-operacije ulaza i izlaza(operacije otvaranja, zatvaranja, čitanja i pisanja)
\item Šta omogućuju multiprocesing i multithreading?\\
-bolje iskorišćenje procesora, istovremena podrška većeg broja korisnika, 
bržu reakciju računara na vanjske dogadjaje
\item Koje operacije uvodi modul za rukovanje procesima?\\
-operacije stvaranja i uništavanja
\item Koje operacije poziva modul za rukovanje procesima?\\
-operacije zauzimanja i oslobadjanja(i operacije stvaranja i uništavanja)
\item Koje module sadrži slojeviti operativni sistem?\\
- modul za rukovanje procesima, modul za rukovanje datotekama, modul za 
rukovanje radnom memoriom, modul za rukovanje kontrolerima, modul za rukovanje procesorom
\item Šta omogućuju sistemski pozivi?\\
-sistemski pozivi omogućuju prelazak iz korisničkog prostora u sistemski prostor,
radi poziva operacija operativnog sistema
\item Koje adresne prostore podržava operativni sistem?\\
-korisnički prostor (user space) i sistemski prostor (kernel space)
\item Šta karakteriše interpreter komandnog jezika(shell)?\\
-preuzimanje i interpretiranje komandi komandnog jezika
\item Koji nivoi korišćenja operativnog sistema postoje?\\
-programski i interaktivni
\enumerate{}



\end{document}